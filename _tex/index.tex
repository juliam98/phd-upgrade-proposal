% Options for packages loaded elsewhere
\PassOptionsToPackage{unicode}{hyperref}
\PassOptionsToPackage{hyphens}{url}
\PassOptionsToPackage{dvipsnames,svgnames,x11names}{xcolor}
%
\documentclass[
  letterpaper,
  DIV=11,
  numbers=noendperiod]{scrartcl}

\usepackage{amsmath,amssymb}
\usepackage{iftex}
\ifPDFTeX
  \usepackage[T1]{fontenc}
  \usepackage[utf8]{inputenc}
  \usepackage{textcomp} % provide euro and other symbols
\else % if luatex or xetex
  \usepackage{unicode-math}
  \defaultfontfeatures{Scale=MatchLowercase}
  \defaultfontfeatures[\rmfamily]{Ligatures=TeX,Scale=1}
\fi
\usepackage{lmodern}
\ifPDFTeX\else  
    % xetex/luatex font selection
\fi
% Use upquote if available, for straight quotes in verbatim environments
\IfFileExists{upquote.sty}{\usepackage{upquote}}{}
\IfFileExists{microtype.sty}{% use microtype if available
  \usepackage[]{microtype}
  \UseMicrotypeSet[protrusion]{basicmath} % disable protrusion for tt fonts
}{}
\makeatletter
\@ifundefined{KOMAClassName}{% if non-KOMA class
  \IfFileExists{parskip.sty}{%
    \usepackage{parskip}
  }{% else
    \setlength{\parindent}{0pt}
    \setlength{\parskip}{6pt plus 2pt minus 1pt}}
}{% if KOMA class
  \KOMAoptions{parskip=half}}
\makeatother
\usepackage{xcolor}
\setlength{\emergencystretch}{3em} % prevent overfull lines
\setcounter{secnumdepth}{-\maxdimen} % remove section numbering
% Make \paragraph and \subparagraph free-standing
\makeatletter
\ifx\paragraph\undefined\else
  \let\oldparagraph\paragraph
  \renewcommand{\paragraph}{
    \@ifstar
      \xxxParagraphStar
      \xxxParagraphNoStar
  }
  \newcommand{\xxxParagraphStar}[1]{\oldparagraph*{#1}\mbox{}}
  \newcommand{\xxxParagraphNoStar}[1]{\oldparagraph{#1}\mbox{}}
\fi
\ifx\subparagraph\undefined\else
  \let\oldsubparagraph\subparagraph
  \renewcommand{\subparagraph}{
    \@ifstar
      \xxxSubParagraphStar
      \xxxSubParagraphNoStar
  }
  \newcommand{\xxxSubParagraphStar}[1]{\oldsubparagraph*{#1}\mbox{}}
  \newcommand{\xxxSubParagraphNoStar}[1]{\oldsubparagraph{#1}\mbox{}}
\fi
\makeatother


\providecommand{\tightlist}{%
  \setlength{\itemsep}{0pt}\setlength{\parskip}{0pt}}\usepackage{longtable,booktabs,array}
\usepackage{calc} % for calculating minipage widths
% Correct order of tables after \paragraph or \subparagraph
\usepackage{etoolbox}
\makeatletter
\patchcmd\longtable{\par}{\if@noskipsec\mbox{}\fi\par}{}{}
\makeatother
% Allow footnotes in longtable head/foot
\IfFileExists{footnotehyper.sty}{\usepackage{footnotehyper}}{\usepackage{footnote}}
\makesavenoteenv{longtable}
\usepackage{graphicx}
\makeatletter
\def\maxwidth{\ifdim\Gin@nat@width>\linewidth\linewidth\else\Gin@nat@width\fi}
\def\maxheight{\ifdim\Gin@nat@height>\textheight\textheight\else\Gin@nat@height\fi}
\makeatother
% Scale images if necessary, so that they will not overflow the page
% margins by default, and it is still possible to overwrite the defaults
% using explicit options in \includegraphics[width, height, ...]{}
\setkeys{Gin}{width=\maxwidth,height=\maxheight,keepaspectratio}
% Set default figure placement to htbp
\makeatletter
\def\fps@figure{htbp}
\makeatother

\usepackage{pgfgantt}
\usepackage[dvipsnames]{xcolor}
\definecolor{myblue}{rgb}{0.592, 0.737, 0.878}
\usepackage[acronym, nonumberlist, nopostdot, hyperfirst=false]{glossaries}
\usepackage{glossary-mcols}
\loadglsentries{files/abbreviations}
\setkomafont{title}{\normalfont\large\color{myblue}}
\setkomafont{section}{\normalfont\large\color{myblue}}
\setkomafont{subsection}{\normalfont\normalsize\color{myblue}}
\KOMAoption{captions}{tableheading}
\makeatletter
\@ifpackageloaded{caption}{}{\usepackage{caption}}
\AtBeginDocument{%
\ifdefined\contentsname
  \renewcommand*\contentsname{Table of contents}
\else
  \newcommand\contentsname{Table of contents}
\fi
\ifdefined\listfigurename
  \renewcommand*\listfigurename{List of Figures}
\else
  \newcommand\listfigurename{List of Figures}
\fi
\ifdefined\listtablename
  \renewcommand*\listtablename{List of Tables}
\else
  \newcommand\listtablename{List of Tables}
\fi
\ifdefined\figurename
  \renewcommand*\figurename{Figure}
\else
  \newcommand\figurename{Figure}
\fi
\ifdefined\tablename
  \renewcommand*\tablename{Table}
\else
  \newcommand\tablename{Table}
\fi
}
\@ifpackageloaded{float}{}{\usepackage{float}}
\floatstyle{ruled}
\@ifundefined{c@chapter}{\newfloat{codelisting}{h}{lop}}{\newfloat{codelisting}{h}{lop}[chapter]}
\floatname{codelisting}{Listing}
\newcommand*\listoflistings{\listof{codelisting}{List of Listings}}
\makeatother
\makeatletter
\makeatother
\makeatletter
\@ifpackageloaded{caption}{}{\usepackage{caption}}
\@ifpackageloaded{subcaption}{}{\usepackage{subcaption}}
\makeatother

\ifLuaTeX
  \usepackage{selnolig}  % disable illegal ligatures
\fi
\usepackage[style=nature,]{biblatex}
\addbibresource{My Library.bib}
\usepackage{bookmark}

\IfFileExists{xurl.sty}{\usepackage{xurl}}{} % add URL line breaks if available
\urlstyle{same} % disable monospaced font for URLs
\hypersetup{
  pdftitle={Upgrade Proposal},
  colorlinks=true,
  linkcolor={blue},
  filecolor={Maroon},
  citecolor={Blue},
  urlcolor={Blue},
  pdfcreator={LaTeX via pandoc}}


\title{Upgrade Proposal}
\author{Julia Marcinkowska}
\date{}

\begin{document}
\maketitle


\section{Background}\label{background}

\emph{Summary of current state of the field and context within which the
research is located, covering current theory/state of the evidence and
referring to relevant literature (500-1,000 words).}

The {NMDA} hypofunction hypothesis of schizophrenia proposes that
decreased activity of {NMDA} receptors has a key role in the development
of schizophrenia pathology. The affected {NMDA} receptors are primarily
localised at {GABA}-ergic fast-spiking {PV} interneurons; where
decreased activity of {PV} interneurons causes a disinhibition of their
activity on pyramidal neurons, disrupting the {EI} balance, and leading
to increased excitation. Hyperactivity in the hippocampus is observed in
the early stages in schizophrenia, as well as in people at clinical high
risk of schizophrenia that subsequently develop the disorders,
suggesting this region might be implicated in the development of the
pathology at early stages of the disorder. This is consistent with the
observations that administation of {NMDA} antagonists like phencyclidine
and ketamine induces behaviours comparable to all three schizoprenia
symptom dimensions (positive, negative, and cognitive symptoms)
(citations from \autocite{nakazawa_origin_2020}), and repeated
administration results in increased release of {DA} in rodent striatum
citations from \autocite{nakazawa_origin_2020}, suggesting that
hyperdopaminergia is caused by decreased {NMDA} activity
\autocite{grace_dopamine_2012,grace_dysregulation_2016}.

Alterations in synaptic function have also been implicated in the
aetiology of schizophrenia \autocite{howes_synaptic_2023}.
Excitotoxicity caused by increased glutamatergic activity might be one
of the contributing factors in the reduction in synaptic connections in
schizophrenia. Postmortem studies have reported decreased
\autocite{abdallah_effects_2018}

\section{Aims and objectives}\label{aims-and-objectives}

\begin{itemize}
\tightlist
\item
  The aim of my project is to examine the relationship between synaptic
  connectivity and glutamatergic function. To do this I will measure the
  difference in glutamate levels (MRS) after administration of LEV and
  placebo in healthy controls and people with schizophrenia.
\item
  The recruitment target is 50 participants: 25 healthy controls (HC)
  and 25 people with schizophrenia (SZ).
\end{itemize}

\section{Hypotheses under
investigation}\label{hypotheses-under-investigation}

I will aim to answer the following questions:

\begin{enumerate}
\def\labelenumi{\arabic{enumi}.}
\tightlist
\item
  Does modulating SV2A lead to lower glutamate levels in healthy people?
\item
  Does modulating SV2A lead to lower glutamate levels in people with
  schizophrenia? Is the change different to that in healthy controls?
\item
  Does modulating SV2A lead to change in symptoms in schizophrenia?
\end{enumerate}

\section{Methodology}\label{methodology}

\subsection{Study design and data
collection}\label{study-design-and-data-collection}

\begin{itemize}
\tightlist
\item
  Single-blind, randomised, placebo-controlled trial with cross-over
  design.
\end{itemize}

\subsection{Analysis}\label{analysis}

\begin{itemize}
\tightlist
\item
  MRS data processing with Osprey - Hippocampus and ACC SVS
\end{itemize}

\subsection{Statistics}\label{statistics}

\begin{itemize}
\tightlist
\item
  Statistical analysis: ANOVA / t-test (within and between group). Power
  calculations.
\item
  I will compare the effect of levetiracetam on Glx levels in healthy
  controls (HC) and patients with schizophrenia (SZ). This will be
  visualised on a raincloud plot such as the one below. The data used in
  this graph is made up for the purpose of illustrating the types of
  comparisons and data visualisations I plan on using.
\end{itemize}

\begin{figure}[H]

\centering{

\includegraphics{index_files/figure-latex/notebooks-plots-fig-lev_hc_vs_sz-output-2.png}

}

\caption{\label{fig-lev_hc_vs_sz}Comparison of GLX levels change between
placebo and levetiracetam in HC and SZ}

\end{figure}%

\textsubscript{Source:
\href{https://juliam98.github.io/phd-upgrade-proposal/notebooks/plots-preview.html\#cell-fig-lev_hc_vs_sz}{Plots}}

\section{Progress made to date, including pilot work, if
applicable}\label{progress-made-to-date-including-pilot-work-if-applicable}

\section{Planned future work}\label{planned-future-work}

\section{Contribution to existing
knowledge.}\label{contribution-to-existing-knowledge.}

\textbf{How the research will form a distinct contribution to existing
knowledge on the subject and afford evidence of originality shown by
discovery of new facts or exercise of independent critical power}

\section{Personal share in
investigations}\label{personal-share-in-investigations}

\textbf{Where work is done in conjunction with the supervisor and/or
with collaborators or other students, a statement of the candidate's own
personal share in the investigations}

\section{Timeline for the remainder of
studies.}\label{timeline-for-the-remainder-of-studies.}

\begin{figure}[H]

\centering{

\includegraphics{index_files/figure-latex/notebooks-plots-fig-gantt-chart-output-1.png}

}

\caption{\label{fig-gantt-chart}Gantt chart of planned work during my
PhD}

\end{figure}%

\textsubscript{Source:
\href{https://juliam98.github.io/phd-upgrade-proposal/notebooks/plots-preview.html\#cell-fig-gantt-chart}{Plots}}

\newpage{}

\section{References}\label{references}

\printbibliography[heading=none]





\end{document}
